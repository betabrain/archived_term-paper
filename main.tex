\documentclass[a4paper,man,12pt,apacite]{apa6} % man => jou
\usepackage[english]{babel}
\usepackage[utf8]{inputenc}
\usepackage{mathtools}
\usepackage{url}

\begin{document}

\title{Random Forests Destructured: Introduction, Overview, Possibilities}
\shorttitle{Random Forests Destructured}
\author{Tobias Ammann}
\keywords{ensemble methods, introduction, random forests}
\affiliation{Literature Study at the Workgroup for Psychological Methods, Evaluation and Statistics, Department of Psychology.\\Supervised by Prof. Dr. Carolin Strobl}

\abstract{This report discusses a rather new machine learning algorithm called
random forest, it's qualities, use, problems, and a small number of
improvements that have been tried.
Random forests are getting a lot of attention outside of psychology
at the moment, and it would be nice to spread that fever within psychology too.
However, it is important to keep in mind that random forests are
relatively unstudied. This hasn't hindered a number of people to suggest
improvements. This paper introduces the reader to frandom forests, and
tries to give the reader a more practically guided understanding of the
method, which hopefully leads to a better use of random forests.}

\keywords{ensemble methods, introduction, random forests}

\maketitle

\tableofcontents

\section{Introduction}

\subsection{Motivation}
Modern psychological research has to provide more than the interesting
case studies and adventurous theories, that psychology consisted of when
Freud was alive.
Psychology has become a science, thus psychological research has to follow
the \emph{scientific method}, according to which positive proof is an
impossibility unless we have complete knowledge, and could eliminate all
alternative theories.
Since we won't ever have complete knowledge, this isn't what
research is all about.
Research works under the assumption, that if we disprove just enough
alternative theories, we can eventually tell which theory is probably
true.
So, the scientific method really is nothing but the use of spades of
attempts to disprove alternative theories, until only a single such theory
remains.

Since this is an unweildly number of theories to disprove, and every
researcher likes to see the result of his work during his lifetime, a
more speedy method is usually employed, although this comes with a caveat.
The speedier method has the researcher pit his favored theory against
the null hypothesis, a fancy word for chance.
This certainly is way more efficient than comparing the thousands of
theories that researchers have come up with, and continue to come up with.
The caveat, commonly called confirmation bias, is that the result only
has significance in the experimental setup being tested.
On the greater scale of things, i.e. reality, the results might well be
completely bogus.
Thankfully, most of the things being studied are constant or change
predictably enough to allow researchers to generalize from the results
of an experiment to the world at large, and likely remain correct.
This likelihood depends on the size of the effects measured in the
experiment, the number of measurements, and on the properties of the
statistical methods involved.
In psychological research, where large effects are freakishly rare and
experiments usually study only a handful of psychology students, it is
vital to use good statistical methods, because that is the only parameter
remaining for the researcher to tweak in his favour.

Recently, researchers in psychology began to turn to a new breed of
statistical methods, in hope for ever better results. This new breed of
statistical methods is called \emph{machine learning}.

In this paper, I aim to introduce the reader to \emph{random forests},
which are just one family of machine learning algorithms.
I intend to do this in a way that gives every reader a chance to understand
this method without prior knowledge. I also intend to present the reader
with some context around random forests, in hope that they will benefit
from a more big-picture view.

In the next sections I am going to introduce the reader to machine learning,
the differences between the traditional statistical and this new
machine learning approach, random forests in particular, and finally delve
into some improvements to random forests that have been suggested in the
literature.

\subsection{Machine Learning}
In order to understand random forests, it might be useful to set
the stage by shortly discussing machine learning in general.
Machine Learning is both a part of predictive statistics and the
artificial intelligence branch of computer science.

\emph{Predictive statistics} is the subfield of statistics that is
concerned with making predictions based on past observations.
It's probably most widely known method is \emph{linear regression}
\cite{wpLR} that associates two variables \(y\) and \(x\) in such a way
that they describe a straight line: \(y = \alpha * x + \beta \).
Predictive statistics is widely used in psychology because it
allows the researcher to look at the unobservable by making
assumptions of the form \(reaction = mind * stimulus + variation\).
This is the standard approach in personality questionaires.

\emph{Artificial intelligence} is a field commonly associated with the
computer sciences, where it began with the advent of higher-order
programming languages based on mathematical foundations around the \emph{1960-ies} \cite{wpHOPL}.
It's aim is to give computers human-like capabilites, so that they can assist
us by combining intelligence, with perfect logic and super-human knowledge.
It includes things like \emph{logic programming} \cite{wpLP},
\emph{expert systems} \cite{wpES}, \emph{databases} \cite{wpDB} and
\emph{neural networks} \cite{wpNN}, that all represent some form of
storing and querying knowledge.
Unfortunately, computers back then didn't have the speed and memory required
to push the envelope far enough, and the field was deemed dead.
Only the rather recent coexistance of powerful computers and massive amounts
of stored data revived artificial intelligence as an important field of
research.
\emph{Machine learning} is that part of artificial intelligence that
is concerned with the automatic learning of facts about the world in order
that they can be stored and subsequently queried later on.
As such it is concernded with making statements based on past observations,
and therefore close to predictive statistics \cite{wpML}.

\subsection{The Machine Learning Life Cycle}
Here I'll discuss the difference between machine learning and traditional
statistical methods. Terminology.
The following section heavily relies on information found in Wikipedia, as
well as what I learned during statistics lectures in the past years.

Traditionally, the statistical methods that are being used in psychology
take a model of how the world works, and a bunch of data, and return one of
two things.
The methods called \emph{regression methods} try to derive the values of
one variable from the other variables in the dataset using the formula the
researcher suggested. They then check how much better the predicted values
fit the actual values than the mean value of the actual values.
Finally, the method outputs the probability with which this improvement in
prediction could happen by chance.
Based on \cite{wpRA}.
The methods called \emph{analysis of variance methods} partition the dataset
according to all but one variable. They then calculate the probability with
which the variation in the one variable among the groups could be due to
random variations in the dataset.
Based on \cite{wpAOV}.
These probabilities that the methods output are what statisticians call the
significance. Usually they define a target significance level, e.g. 5\%, and
compare it to the output of their statistical calculations. If the
calculated probability is less than the targeted significance level the
measured effects are said to he significant at the chosen significance level.

The most striking difference between traditional statistical methods and
machine learning methods is, that the researcher can't specify his model of
how the world works, other than through the selection of machine learning
algorithm.
Because of that, machine learning algorithms are sometimes described as
\emph{black boxes}, meaning that the user can only see what's going into the
algorithm, and what is coming out.
This is, because machine learning algorithms derive the model on their own.
This is what the learning in machine learning means.
The second difference is what the algorithms return.
Since machine learning algorithms represent the model, what they output is
not a percentage, i.e. significance, but the predicted values.
In short, machine learning provides the researcher with very flexible models
that adapt to the world.
The prediction of such a model can then be compared to test data,
i.e. data that hasn't been included in the learning phase, to calculate a
significance.
The problem with this flexibility is, that one cannot really tell how the
model looks like, that is, the model is not in a human-readable form.
As will be pointed out later, decision trees, the underlying mechanism in
random forests, is quite easily interpretable, but since random forests
consist of dozens to thousands of such trees, a human can hardly tell what
they mean.
Therefore, it is very important to find ways to condense this complexity
into something that can be more easily interpreted.
The variable importance measure of random forests is one such way.
Based on \cite{wpML}.

\subsection{Classification}
The \emph{classification problem} is the problem classifying new data based
on a set of example data, but without the explicit ruleset that guided its
classification.
Unlike in \emph{regression}, the result here is one of many given classes
and not a value.
One might describe classification as regression with discreet output values.
Output variables are commonly called \emph{classes}

\subsection{Regression}
A short discussion of regression, the why and how, and the difference
between classical regression and regression in machine learning.
Terminology. Classification with continuous classes.

\subsection{Randomness}
Many machine learning algorithms use random numbers in different places.
While computer generated random numbers are not truly random, they still
cause the outcome of the algorithm to change slightly between different
executions.
Which might unnerve a novice, but doesn't change the outcome of the
algorithm significantly.
Still, for publication purposes it can make sense to set and publish the
random number generator's \emph{seed}, though doing so during the experiment
this should never be done.

\section{Random Forests}

\subsection{Ensemble Methods}
A short introduction to ensembles and their properties. Terminology.
Weak and strong learners.

\subsubsection{Bootstrapping}
Sampling the data.

\subsubsection{Random Subspaces}
Sampling the variables.

\subsubsection{Boosting}
Weighing the sample.

\subsection{Trees and forests}
The \emph{trees} computer scientists refer to are artificial constructs that
branch out starting from a single point, called \emph{root node}.
A common \emph{fan-out factor} is two, meaning that every node in the tree
as two branches connecting it to two other nodes.
Nodes without \emph{child nodes} of their own are called \emph{leaf nodes}.
Keeping with the reference to nature, the word \emph{forest} denotes a
collection of trees.

\subsubsection{Decision Trees}
A \emph{decision tree} is a tree that sketches a decision process: every
node stands for a criterion, and leaf nodes denote outcomes.
To use a decision tree, one starts at the root node, and follows the path
along nodes whose criteria one affirms to a leaf node, i.e. the outcome of
the decision process.

Here is a short example to clarify the idea of decision trees, to decode
on where one should eat lunch.
The first criterion is hunger, hence we let the root node to encode that
\( hunger > snack \), and decode against eating if this is false and abort.
If we are hungry however, we proceed to the next node \( cash > price \).
We decide to eat if we can afford it, and abort if we can't.

The advantage of decision trees is, that they can be constructed by an
algorithm, but still provide meaningful information to a human reader.

\subsubsection{Random Forests}
Random forests were invented by Leo Breiman.
The following description is based on his original publication in the
journal \emph{Machine Learning} in 2001 \cite{breiman2001random}.

\emph{Learning and Predicting}

\emph{Variable Importance}

\subsection{In Search of the Super-Tree}
The fact that the random forest method is still being studied, that its
optimal parameter settings are still being debated, and that it's results
are not always better than other methods, led to the development of
variants.
Every variant tries to address a specific problem, tries to increase the
learning efficiency or tries to better prevent overfitting.

\subsubsection{Alternative: Dynamic Random Forests}
Boosting random forests.

\subsubsection{Alternative: PERT}
Random variables and random splits.

\subsubsection{Alternative: Rotation Forest}
Massaging features before growing trees.

\subsubsection{Alternative: Fuzzy Random Forests}

\section{Method}

\subsection{Selection of Papers}
I have started my research on \emph{random forests} by reading the
introductory paper suggested by my supervisor titled
\emph{An introduction to recursive partitioning: Rationale, application
and characteristics of classification and regression trees, bagging and
random forests} \cite{strobl2009introduction}.
I then searched the research databases \emph{PsychARTICLES} and
\emph{PsychINFO} querying for \emph{random forest} and limiting my results
to the last two years.
I only considered papers that focuss on \emph{random forests} and
dismissed every paper where \emph{random forests} are merely used as a
research method.
I also used \emph{Google Scholar} to look for more technical
publications outside the field of psychology.
I searched for combinations of \emph{random forest}, \emph{comparison},
\emph{analysis}, and \emph{ensemble}.
I also included keywords seen in interesting titles, like \emph{fuzzy},
\emph{perfect}, \emph{full}, \emph{balanced}, \emph{extremely}, and
\emph{rotation}.
I also queried for some of the referenced publications while I read the
found material, but only included \cite{strobl2008conditional}, because
it was referenced multiple times, and co-authored by my supervisor.

The final criteria for inclusion were the online availability of a freely
downloadable PDF-file, which thanks to \emph{Google Scholar} often turned
out to be no problem at all, and my decision on the topic of
this report.

A lot of the information in the fields of computer science, artificial
intelligence, machine learning, databases, and especially programming
has been acquired through different means in the last ten years.
Since I can't remember the original sources, I took to include the pages
on Wikipedia, which I used to refresh my memories.

\subsection{Aim and Structure of the Paper}
I had a couple of obvious possibilities of where to take this topic, which
was given to me in form of the above mentioned introductory paper.
These possibilities were the presentation of examplary uses of random
forests, the discussion of strengths and weaknesses, and any of the more
specialized variants.
During my research I got the impression, that there were serious
differences in the understanding of random forests among researchers,
and even among designers of improved variants.
A study comparing different decision tree ensemble techniques also pointed
this out by saying that many of the commonly used methods of comparison
weren't robust enough for use with random forests \cite{banfield2007comparison}.

Because of this fuzziness, I decided for a different focus, and to write
a very broad - big picture - introduction to random forests.

\section{Conclusion}
State of the art? Well... They are not yet properly understood, and
many comparisons and lots of the advantages might be accidential.
Strong dependance on parameters, with no rationally pleasing way to set them.

\subsection{The End}
Bye.

\bibliography{references}

\end{document}

