\documentclass[a4paper,jou,12pt,apacite]{apa6}
\usepackage[english]{babel}
\usepackage[utf8]{inputenc}

\begin{document}

\title{Random Forests Destructured: Introduction, Overview, Possibilities}
\shorttitle{Random Forests Destructured}
\author{Tobias Ammann}
\keywords{ensemble methods, introduction, random forests}
\affiliation{Literature Study at the Workgroup for Psychological Methods, Evaluation and Statistics, Department of Psychology.\\Supervised by Prof. Dr. Carolin Strobl}

\abstract{Random forest are a machine learning technique that is getting some attention in psychological research lately. This paper will give an introduction to random forests and discuss some vital points.}
\keywords{ensemble methods, introduction, random forests}

\maketitle

\tableofcontents

\section{Introduction}

\subsection{Motivation}
Hi there, this is a test. Today is the \today, and I love it \cite{liu2005maximizing}. ...there is so much more\footnote{And more.}. Of course, let's
not forget about the awesome work done in \cite{cutler2001pert}.

\subsection{Potential Benefits}

\section{Random Forests}

\subsection{Machine Learning}

\subsubsection{Life Cycle}

\subsubsection{Classification}

\subsubsection{Regression}

\subsection{Ensembles}

\subsection{Classifier}

\subsection{Parameters}

\subsection{Randomness}

\section{Method}

\subsection{Selection of Papers}

\section{Conclusion}

\subsection{The End}
Bye.

\bibliography{references}

\end{document}

